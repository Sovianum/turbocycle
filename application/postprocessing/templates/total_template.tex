\documentclass{article}
\usepackage{mathtext}
\usepackage[T2A]{fontenc}
\usepackage[utf8]{inputenc}
\usepackage[russian]{babel}
\usepackage{amsmath}
\usepackage{amsfonts}
\usepackage{amssymb}
\usepackage{graphicx}
\usepackage{geometry}
\usepackage{color}
\usepackage{gensymb}

\usepackage{enumitem}
\setlist[enumerate]{label*=\arabic*.}

\usepackage{indentfirst}

\usepackage{titlesec}
\newcommand{\sectionbreak}{\clearpage}

\graphicspath{ {<GraphicsPath>/} }

\begin{document}
\begin{titlepage}
  \begin{center}
    МОСКОВСКИЙ ГОСУДАРСТВЕННЫЙ УНИВЕРСИТЕТ \\ ИМ. Н.Э.БАУМАНА
    \vspace{0.25cm}

    Факультет "Энергомашиностроение"

    Кафедра "Э3"
    \vfill


    Клюквин Артём Дмитриевич
    \vfill

    \textsc{Курсовоей проект}\\[5mm]

    {\LARGE Проектирование турбины транспортного ГТД}
\end{center}
\vfill

\newlength{\ML}
\settowidth{\ML}{«\underline{\hspace{0.7cm}}» \underline{\hspace{2cm}}}
\hfill\begin{minipage}{0.4\textwidth}
  Руководитель курсового проекта\\
  \underline{\hspace{\ML}} Н.\,И.~Троицкий\\
  «\underline{\hspace{0.7cm}}» \underline{\hspace{2cm}} 2016 г.
\end{minipage}%
\bigskip

\vfill

\begin{center}
  Москва, 2016 г.
\end{center}
\end{titlepage}



\subsection{Вариантные расчеты}
Для определения оптимальной степени повышения давления в компрессоре построим графики зависимости КПД, удельной мощности и расхода через компрессор от степени повышения давления в компрессоре. При этом для наглядности отнесем абсолютные значения рассматриваемых величин к максимальному значению, достигающемуся на заданном промежутке.

Ниже представлен график зависимостей КПД, мощности и расхода ГТД. Для удобства значения переменных на графике отнесены к максимальному значению достигающемуся на рассматриваемом диапазоне изменения степени повышения давления компрессора:
\begin{center}
	\includegraphics[scale=0.5]{cycle}
\end{center}

Экстремум по КПД достигается при следующих значения функций:
\begin{center}
	\begin{tabular}{|c|c|c|c|}
	\hline
	$G, кг/с$ & $N_e, Вт/кг$ & $\eta_e$ & $\pi_к$ \\ \hline
	$<EtaMaxG>$ & $<EtaMaxNe> \cdot 10^6$ & $<EtaMaxEta>$ & $<EtaMaxPi>$ \\ \hline
	\end{tabular}
\end{center}

Экстремум по удельной мощности достигается при следующих значениях функций:
\begin{center}
	\begin{tabular}{|c|c|c|c|}
	\hline
	$G, кг/с$ & $N_e, Вт/кг$ & $\eta_e$ & $\pi_к$ \\ \hline
	$<NeMaxG>$ & $<NeMaxNe> \cdot 10^6$ & $<NeMaxEta>$ & $<NeMaxPi>$ \\ \hline
	\end{tabular}
\end{center}

Несмотря на то, что двигатель газотрубовоза значительную часть времени работает на режимах частичной мощности, принимается $\pi_к = <PiComp>$ , что позволяет сделать турбину компрессора одноступенчатой.

Ниже представлен расчет цикла ГТД при $\pi_к = <PiComp>$



\end{document}