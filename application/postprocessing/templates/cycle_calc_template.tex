\subsection{Расчет цикла при $\pi_{КНД} = <-<.LPCompressor.Pi>->, \pi_{КВД} = <-<.HPCompressor.Pi>->$}
% Расчет ГТД носит итерационный характер, так как параметры газа на входе в камеру сгорания
% за счет регенератора зависят от температуры на выходе из силовой турбины.
% Таким образом, расчет, представленный ниже, относится к последней итерации для
% случая $\pi_к = <PiComp>$. При этом принимается температура на выходе из силовой
% турбины: $T_т = <FreeTurbineTOut>$.
\begin{enumerate}
	\item Оределим давление за входным фильтром:
		$$p_ф=\sigma_ф p_a = <-<.InletFilter.Sigma>-> \cdot <-<.GasSource.P>-> = <-<.InletFilter.POut>-> \/\ МПа$$
	\item Определим давление за входным устройством:
		$$p_{вх} = \sigma_{вх} p_{вх} = <-<.InletPipe.Sigma>-> \cdot <-<.InletPipe.Sigma>-> = <-<.InletPipe.POut>-> \/\ МПа$$
	




	\item Определим давление за КНД:
		$$p_{КНД} = \pi_к p_{вх} = <-<.LPCompressor.PIn>-> \cdot <-<.LPCompressor.Pi>-> = <-<.LPCompressor.POut>-> \/\ МПа$$
	\item Определим температуру газа за КНД, принимая показатель адиабаты воздуха $k_{в \/\ КНД} = <-<.LPCompressor.GasData.KMean>->$:
		$$T_{КНД} = T_a 
		\left[ 
			1 + \frac{
				\pi_к^{
					\frac{
						k_{в \/\ КНД} - 1
					}{
						k_{в \/\ КНД}
					}
				} - 1
			}{
				\eta_к
			}
		\right] =
			<-<.LPCompressor.TIn>-> 
		\left[
			1 + \frac{
				{<-<.LPCompressor.Pi>->}^{
					\frac{
						<-<.LPCompressor.GasData.KMean>-> - 1
					}{
						<-<.LPCompressor.GasData.KMean>->
					}
				} - 1
			}{
				<-<.LPCompressor.Eta>->
			}
		\right] = <-<.LPCompressor.TOut>-> \/\ К$$
	\item Используя найденный показатель адиабаты воздуха, определим теплоемкость воздуха в процессе сжатия воздуха в КНД:
		$$c_{pв \/\ КНД} = \frac{
			k_{в \/\ КНД}
		}{
			k_{в \/\ КНД} - 1
		} R_в = \frac{
			<-<.LPCompressor.GasData.KMean>->
		}{
			<-<.LPCompressor.GasData.KMean>-> - 1
		} \cdot <-<.LPCompressor.GasData.R>-> = <-<.LPCompressor.GasData.CpMean>-> \/\ Дж/кг$$
	\item Определим работу КНД:
		$$L_{КНД} = c_{pв \/\ КНД} \left( T_{КНД} - T_a \right) =
			<-<.LPCompressor.GasData.CpMean>-> \cdot \left(<-<.LPCompressor.TOut>-> - <-<.LPCompressor.TIn>->\right) =
			<-<.LPCompressor.Labour>-> \cdot 10^6 \/\ Дж/кг = $$
	



	\item Определим давление перед КВД:
		$$p_{0 \/\ КВД} = \sigma_{КНД} p_{КНД} = <-<.LPCompressorPipe.Sigma>-> \cdot <-<.LPCompressor.POut>-> = <-<.HPCompressor.PIn>-> \/\ МПа$$




	\item Определим давление за КВД:
		$$p_{КВД} = \pi_{КВД} p_{0 \/\ КВД = <-<.HPCompressor.PIn>-> \cdot <-<.HPCompressor.Pi>-> = <-<.HPCompressor.POut>-> \/\ МПа$$
	\item Определим температуру газа за КВД, принимая показатель адиабаты воздуха $k_{в \/\ КВД} = <-<.HPCompressor.GasData.KMean>->$:
		$$T_{КВД} = T_{КНД}
		\left[ 
			1 + \frac{
				\pi_к^{
					\frac{
						k_в - 1
					}{
						k_в
					}
				} - 1
			}{
				\eta_к
			}
		\right] =
			<-<.HPCompressor.TIn>-> 
		\left[
			1 + \frac{
				{<-<.HPCompressor.Pi>->}^{
					\frac{
						<-<.HPCompressor.GasData.KMean>-> - 1
					}{
						<-<.HPCompressor.GasData.KMean>->
					}
				} - 1
			}{
				<-<.HPCompressor.Eta>->
			}
		\right] = <-<.HPCompressor.TOut>-> \/\ К$$
	\item Используя найденный показатель адиабаты воздуха, определим теплоемкость воздуха в процессе сжатия воздуха в КВД:
		$$c_{pв \/\ КВД} = \frac{
			k_{в \/\ КВД}
		}{
			k_{в \/\ КВД} - 1
		} R_в = \frac{
			<-<.HPCompressor.GasData.KMean>->
		}{
			<-<.HPCompressor.GasData.KMean>-> - 1
		} \cdot <-<.HPCompressor.GasData.R>-> = <-<.HPCompressor.GasData.CpMean>-> \/\ Дж/кг$$
	\item Определим работу КВД:
		$$L_{КВД} = c_{pв \/\ КВД} \left( T_{КВД} - T_{КНД} \right) =
			<-<.HPCompressor.GasData.CpMean>-> \cdot \left(<-<.HPCompressor.TOut>-> - <-<.HPCompressor.TIn>->\right) =
			<-<.HPCompressor.Labour>-> \cdot 10^6 \/\ Дж/кг = $$




	\item Температура газа за камерой сгорания:
		$$T_г = <-<.Burner.Tg>-> \/\ К$$
	\item Определим относительный расход топлива. Расчет носит итерационный характер. Ниже описана последняя итерация. Теплоемкость продуктов сгорания природного газа рассчитывается через показатель адиабаты и газовую постоянную газа. При этом газовая постоянная и истинный показатель адиабаты рассчитываются как средневзвешенное соответственных характеристик компонентов продуктов). При расчета приняты следующие значения:
	\begin{enumerate} % список значений для расчета удельного расхода топлива
		\item[1)] теплоемкость топлива:
			$$c_{pm} = <-<.Burner.Fuel.C>-> \/\ Дж / (кг \cdot К);$$
		\item[2)] температура подачи топлива:
			$$T_m = <-<.Burner.Fuel.TInit>-> \/\ К;$$
		\item[3)] температура определения теплофизических параметров веществ:
			$$T_0 = <-<.Burner.Fuel.T0>-> \/\ К;$$
		\item[4)] истинная теплоемкость воздуха перед камерой сгорания:
			$$c_{pв \/\ г}\left( T_{КВД} \right) = <-<.Burner.AirDataInlet.Cp>-> \/\ Дж/(кг \cdot К);$$
		\item[5)] истинная теплоемкость воздуха при температуре определения теплофизических параметров веществ:
			$$c_{pв \/\ г}\left( T_0 \right) = <-<.Burner.AirData0.Cp>-> \/\ Дж/(кг \cdot К);$$
		\item[6)] низшая теплота сгорания топлива:
			$$Q_н^р = <-<.Burner.Fuel.QLower>-> \cdot 10^3 \/\ Дж / (кг \cdot К);$$
		\item[7)] полнота сгорания:
			$$\eta_г = <-<.Burner.Eta>->;$$
		\item[8)] масса воздуха, необходимая для сжигания 1 кг топлива:
			$$l_0 = <-<.Burner.Fuel.L0>-> \/\ кг;$$
	\end{enumerate}

	\begin{enumerate}
		\item Зададимся коэффициентом избытка воздуха: $$\alpha = <-<.Burner.Alpha>->;$$
		\item Теплоемкость продуктов сгорания природного газа $c_{pг \/\ г}$ при данном значении коэффициента избытка воздуха при температуре $T_г$ составляет:
			$$c_{pг \/\ г}\left( T_г \right) = <-<.Burner.GasDataOutlet.Cp>-> \/\ Дж/(кг \cdot К);$$
		\item Теплоемкость продуктов сгорания природного газа $c_{pг \/\ г}$ при данном значении коэффициента избытка воздуха при температуре $T_0$ составляет:
			$$c_{pг \/\ г}\left( T_0 \right) = <-<.Burner.GasData0.Cp>-> \/\ Дж / (кг \cdot К);$$
		\item Определим относительный расход топлива:
			$$g_m = \frac{G_m}{G_в^г} =
		\frac{
			c_{pг \/\ г} \left( T_г \right) T_г -
			c_{pв \/\ г} \left( T_{КВД} \right) T_{КВД} -
			\left[
				\left(
					c_{pг \/\ г}\left( T_0 \right) - c_{pв \/\ г}\left( T_0 \right)
				\right) T_0
			\right]
		}{
			Q_н^р \eta_г -
			\left[
				c_{pг \/\ г} \left( T_г \right) T_г -
				c_{pг \/\ г} \left( T_0 \right) T_0
			\right] +
			c_{pm} \left( T_m - T_0 \right)
		} = $$

		$$=
		\frac{
			<-<.Burner.GasDataOutlet.Cp>-> \cdot <-<.Burner.Tg>-> -
			<-<.Burner.GasDataOutlet.Cp>-> \cdot <-<.HPCompressor.TOut>-> -
			\left[
				\left(
					<-<.Burner.GasData0.Cp>-> - <-<.Burner.AirData0.Cp>->
				\right) \cdot <-<.Burner.AirData0.T>->
			\right]
		}{
			<-<.Burner.Fuel.QLower>-> \cdot 10^3 \cdot <-<.Burner.Eta>-> -
			\left[
				<-<.Burner.GasDataOutlet.Cp>-> \cdot <-<.Burner.Tg>-> -
				<-<.Burner.GasData0.Cp>-> \cdot <-<.Burner.AirData0.T>->
			\right] +
			<-<.Burner.Fuel.C>-> \left( <-<.Burner.Fuel.TInit>-> - <-<.Burner.AirData0.T>-> \right)
		} = <-<.Burner.FuelMassRateRel>->$$
		\item Определим коэффициент избытка воздуха:
			$$\alpha^\prime = \frac{1}{g_m l_0} =
		\frac{1}{<-<.Burner.FuelMassRateRel>-> \cdot <-<.Burner.Fuel.L0>->} = <-<.Burner.Alpha>->$$
	\end{enumerate}

	\item Определим удельный расход через ТВД:
		$$g_{ТВД} = \left( 1 + g_m \right) \left( 1 - g_{ут} - g_{охл} \right) =
			\left( 1 + <-<.Burner.FuelMassRateRel>-> \right) \left( 1 - <-<.HPTurbine.LeakMassRateRel>-> - <-<.HPTurbine.CoolMassRateRel>-> \right) = <-<.HPTurbine.MassRateRel>->$$
	\item Определим удельную работу ТВД:
		$$L_{ТВД} = \frac{L_{КВД}}{g_{ТВД}\eta_{м \/\ ВД}} = \frac{<-<.HPCompressor.Labour>-> \cdot 10^6}{<-<.HPTurbine.MassRateRel>-> \cdot <-<.HPShaft.Eta>-> } = <-<.HPTurbine.Labour>-> \cdot 10^6 \/\ Дж/кг$$
	\item Определим давление газа перед ТВД:
		$$p_г = p_{ТНД} \sigma_г = <-<.HPCompressor.POut>-> \cdot <-<.Burner.Sigma>-> = <-<.HPTurbine.PIn>-> \/\ МПа$$
	\item Определим среднюю теплоемкость газа в процессе расширения газа в турбине, принимая показатель адиабаты газа $k_{г \/\ ТВД} = <-<.HPTurbine.GasData.KMean>->$:
		$$c_{pг \/\ ТВД} = \frac{k_{г \/\ ТВД}}{k_{г \/\ ТВД} - 1} R_г =
			\frac{<-<.HPTurbine.GasData.KMean>->}{<-<.HPTurbine.GasData.KMean>-> - 1} \cdot <-<.HPTurbine.GasData.R>-> = <-<.HPTurbine.GasData.CpMean>-> \/\ Дж/(кг \cdot К) $$
	\item Определим давление воздуха за турбиной компрессора:
		$$p_{ТВД} = p_г
			\left[
				1 - \frac{L_{КВД}}{c_{pг \/\ ТВД} T_г \eta_{ТВД}}
			\right] ^ \frac{k_{г \/\ ТВД}}{k_{г \/\ ТВД} - 1} =
			<-<.HPTurbine.PIn>->
			\left[
				1 - \frac{<-<.HPCompressor.Labour>-> \cdot 10^6}
				{<-<.HPTurbine.GasData.CpMean>-> \cdot <-<.HPTurbine.TIn>-> \cdot <-<.HPTurbine.Eta>->}
			\right] ^ \frac{<-<.HPTurbine.GasData.KMean>->}{<-<.HPTurbine.GasData.KMean>-> - 1} =
			 <CompTurbinePOut> \/\ МПа$$
	\item Определим температуру газа за турбиной компрессора:
	 	$$T_{ТВД} = T_г
			 \left\lbrace
			 	1 -
			 	\left[
			 		1 -
			 			\left(
			 				\frac{p_{ТВД}}{p_г}
			 			\right) ^ \frac{k_{г \/\ ТВД}}{k_{г \/\ ТВД} - 1}
			 	\right] \eta_{ТВД}
			 \right\rbrace =
			 <-<.HPTurbine.TIn>->
			 \left\lbrace
			 	1 -
			 	\left[
			 		1 -
			 			\left(
			 				\frac{<-<.HPTurbine.POut>->}{<-<.HPTurbine.PIn>->}
			 			\right) ^ \frac{<-<.HPTurbine.GasData.KMean>->}{<-<.HPTurbine.GasData.KMean>-> - 1}
			 	\right] \cdot <EtaCompTurb>
			 \right\rbrace = <CompTurbineTOut> \/\ К$$




	\item Определим давление перед ТНД:
		$$p_{0 \/\ ТНД} = p_{ТВД}\sigma_{ТВД} = <-<.HPTurbine.POut>-> \cdot <-<.HPTurbinePipe.Sigma>-> = <-<LPTurbine.PIn>-> \/\ МПа$$



	\item Определим удельный расход через ТВД:
		$$g_{ТВД} = \left( 1 + g_m \right) \left( 1 - g_{ут} - g_{охл} \right) =
			\left( 1 + <-<.Burner.FuelMassRateRel>-> \right) \left( 1 - <-<.HPTurbine.LeakMassRateRel>-> - <-<.HPTurbine.CoolMassRateRel>-> \right) = <-<.HPTurbine.MassRateRel>->$$
	\item Определим удельную работу ТВД:
		$$L_{ТВД} = \frac{L_{КВД}}{g_{ТВД}\eta_{м \/\ ВД}} = \frac{<-<.HPCompressor.Labour>-> \cdot 10^6}{<-<.HPTurbine.MassRateRel>-> \cdot <-<.HPShaft.Eta>-> } = <-<.HPTurbine.Labour>-> \cdot 10^6 \/\ Дж/кг$$
	\item Определим давление газа перед ТВД:
		$$p_г = p_{ТНД} \sigma_г = <-<.HPCompressor.POut>-> \cdot <-<.Burner.Sigma>-> = <-<.HPTurbine.PIn>-> \/\ МПа$$
	\item Определим среднюю теплоемкость газа в процессе расширения газа в турбине, принимая показатель адиабаты газа $k_{г \/\ ТВД} = <-<.HPTurbine.GasData.KMean>->$:
		$$c_{pг \/\ ТВД} = \frac{k_{г \/\ ТВД}}{k_{г \/\ ТВД} - 1} R_г =
			\frac{<-<.HPTurbine.GasData.KMean>->}{<-<.HPTurbine.GasData.KMean>-> - 1} \cdot <-<.HPTurbine.GasData.R>-> = <-<.HPTurbine.GasData.CpMean>-> \/\ Дж/(кг \cdot К) $$
	\item Определим давление воздуха за турбиной компрессора:
		$$p_{ТВД} = p_г
			\left[
				1 - \frac{L_{КВД}}{c_{pг \/\ ТВД} T_г \eta_{ТВД}}
			\right] ^ \frac{k_{г \/\ ТВД}}{k_{г \/\ ТВД} - 1} =
			<-<.HPTurbine.PIn>->
			\left[
				1 - \frac{<-<.HPCompressor.Labour>-> \cdot 10^6}
				{<-<.HPTurbine.GasData.CpMean>-> \cdot <-<.HPTurbine.TIn>-> \cdot <-<.HPTurbine.Eta>->}
			\right] ^ \frac{<-<.HPTurbine.GasData.KMean>->}{<-<.HPTurbine.GasData.KMean>-> - 1} =
			 <CompTurbinePOut> \/\ МПа$$
	\item Определим температуру газа за турбиной компрессора:
	 	$$T_{ТВД} = T_г
			 \left\lbrace
			 	1 -
			 	\left[
			 		1 -
			 			\left(
			 				\frac{p_{ТВД}}{p_г}
			 			\right) ^ \frac{k_{г \/\ ТВД}}{k_{г \/\ ТВД} - 1}
			 	\right] \eta_{ТВД}
			 \right\rbrace =
			 <-<.HPTurbine.TIn>->
			 \left\lbrace
			 	1 -
			 	\left[
			 		1 -
			 			\left(
			 				\frac{<-<.HPTurbine.POut>->}{<-<.HPTurbine.PIn>->}
			 			\right) ^ \frac{<-<.HPTurbine.GasData.KMean>->}{<-<.HPTurbine.GasData.KMean>-> - 1}
			 	\right] \cdot <EtaCompTurb>
			 \right\rbrace = <CompTurbineTOut> \/\ К$$



	
	\item Зададим значение приведенной скорости на выходе из выходного устройства:
		$$\lambda_{вых} = <LambdaOut>$$
	\item Определим давление торможения на выходе из выходного устройства:
		$$p_{вых} = p_a \pi \left( \lambda_{вых}, k_г \right) =
			<AtmP> \cdot \pi \left( <LambdaOut>, <CycleCompTurbineKGasShort> \right) =
			<RegTubePOut> \/\ МПа$$
	\item Определим давление за регенератором по горячей стороне:
		$$p_у = \frac{p_{вых}}{\sigma_{вых}} = \frac{<RegTubePOut>}{<OutletPipeSigma>} =
			<RegenPHotOut> \/\ МПа$$
	\item Определим давление перед регенератором по горячей стороне:
		$$p_{р.г.} = \frac{p_у}{\sigma_{р.г.}} = \frac{<RegenPHotOut>}{<RegenHotSigma>} =
			<RegenPHotIn> \/\ МПа$$
	\item Определим давление торможения за силовой турбиной:
		$$p_т = \frac{p_{р.г.}}{\sigma_т} = \frac{<RegenPHotIn>}{<OutletPipeSigma>} =
	<FreeTurbinePOut> \/\ МПа$$
	\item Определим температуру торможения на выходе из силовой турбины:
		$$T_т = T_{тк}
		 \left\lbrace
		 	1 -
		 	\left[
		 		1 -
		 			\left(
		 				\frac{p_с}{p_т}
		 			\right) ^ \frac{k_г}{k_г - 1}
		 	\right] \eta_т
		 \right\rbrace =
		 <CompTurbineTOut>
		 \left\lbrace
		 	1 -
		 	\left[
		 		1 -
		 			\left(
		 				\frac{<FreeTurbinePIn>}{<FreeTurbinePOut>}
		 			\right) ^ \frac{<CycleCompTurbineKGasShort>}{<CycleCompTurbineKGasShort> - 1}
		 	\right] \cdot <EtaFreeTurb>
		 \right\rbrace = <FreeTurbineTOut> \/\ К$$
	 \item Определим уточненное значение показателя адиабаты газа в процессе расширения в силовой турбине:
	 	$$k_г^\prime = \frac{\int\limits_{T_{тк}}^{T_г} k_г(T) dT}{T_г - T_{тк}} =
			\frac{\int\limits_{<FreeTurbineTOut>}^{<CompTurbineTOut>} k_г(T) dT}{<CompTurbineTOut> - <FreeTurbineTOut>} = <CycleCompTurbineKGasLong>$$
	\item Определим погрешность определения показателя адиабаты газа в процессе расширения в силовой турбине:
		$$\delta = \frac{\left| k_г^\prime - k_г \right|}{k_г} \cdot 100 \% =
			\frac{\left| <CycleCompTurbineKGasLong> - <CycleCompTurbineKGasShort> \right|}{<CycleCompTurbineKGasShort>} \cdot 100 \% =
			<CompTurbineKCalcError> \% < 5 \%$$
	Погрешность определения показателя адиабаты в пределах допуска.
	\item Определим значение теплоемкости газа в свободной турбине, задавая показатель адиабаты газа $k_г = <CycleCompTurbineKGasShort>$:
		$$c_{pг} = \frac{k_г}{k_г - 1} R_г =
			\frac{<CycleCompTurbineKGasShort>}{<CycleCompTurbineKGasShort> - 1} \cdot <CycleCompTurbineGasGasConstant> = <FreeTurbineSpecificHeat> Дж/(кг \cdot К)$$
			\item Определим удельную работу силовой турбины:
		$$L_т = c_{pг} \left( T_{тк} - T_т \right) =
			<FreeTurbineSpecificHeat> \left( <CompTurbineTOut> - <FreeTurbineTOut> \right) =
			<FreeTurbineSpecificLabour> \cdot 10^6 Дж/кг$$
	\item Определим удельную мощность ГТД:
		$$N_{e уд} = L_т g_т =
			<FreeTurbineSpecificLabour> \cdot 10^6 \cdot <FreeTurbineMassRateRel> =
			<EngineSpecificPower> \cdot 10^6 Дж/кг$$
	\item Определим экономичность ГТД:
		$$C_e = \frac{3600}{N_{e уд}} g_т =
			\frac{3600}{<EngineSpecificPower> \cdot 10^6} \cdot <FreeTurbineMassRateRel> =
			<EngineFuelMassRateSpecific> \cdot 10^{-3} Вт/ч$$
	\item Определим КПД ГТД:
		$$\eta_e = \frac{3600}{C_e Q_н^р} =
			\frac{3600}{<EngineFuelMassRateSpecific> \cdot 10^{-3} \cdot <LowerQ> \cdot 10^6}
			= <EngineEta>$$
	\item Определим расход воздуха:
		$$G_в = \frac{N_e}{N_{e уд} \eta_р} =
			\frac{6000 \cdot 10^3}{<EngineSpecificPower> \cdot 10^6 \cdot <EtaGen>} =
			<EngineMassRate> кг/с$$
\end{enumerate}