\section{Поступенчатый расчет турбины}
Для данного проекта выбрана одноступенчатая турбина.
Зададим параметры, необходимые для поступенчатого расчета турбины:

\begin{center}
	\begin{tabular}{|p{7cm}|c|c|c|}
		\hline
		\textbf{Величина} & \textbf{Обозначение} & \textbf{Размерность} & \textbf{Значение} \\ \hline
		Реактивность ступени & $\rho$ & - & $$ \\ \hline
		Радиальный зазор & $\delta_r$ & - & $<DeltaR>$ \\ \hline
		Относительная длина лопатки статора & $\left( \frac{l}{D} \right)_1$ & - & $<StatorLRel>$ \\ \hline
		Удлинение лопатки статора & $\left( \frac{l}{b_a} \right)_{СА}$ & - & $<StatorElongation>$ \\ \hline
		Удлинение лопатки ротора & $\left( \frac{l}{b_a} \right)_{РК}$ & - & $<RotorElongation>$ \\ \hline
		Относительная ширина зазора между лопатками ротора и лопатками статора & $\left( \frac{\delta}{b_a} \right)_{СА}$ & - & $<StatorDeltaRel>$ \\ \hline
		Угол раскрытия на втулке & $\gamma_{вт}$ & \degree & $<GammaIn>$ \\ \hline
		Угол раскрытия на периферии & $\gamma_{пер}$ & \degree & $<GammaOut>$ \\ \hline

	\end{tabular}
\end{center}

\begin{enumerate}
	\item Определим теплоперепад на сопловом аппарате:
	$$H_с = \left( 1 - \rho \right) H_т =
	\left( 1 - <Reactivity> \right) \cdot <Ht> \cdot 10^6 = <Hc> \cdot 10^6 \/\ Дж/кг$$
	\item Определим скорость адиабатного истечения из СА:
	$$c_{1 ад} = \sqrt{2 H_с} = 
	\sqrt{2 <Hc> \cdot 10^6} = <C1ad> \/\ м/с$$
	\item Определим скорость действительного истечения из СА:
	$$c_1 = \phi c_{1 ад} =
	<Phi> \cdot <C1ad> = <C1> \/\ м/с$$
	\item Определим температуру на выходе из СА:
	$$T_1 = T_г - \frac{c_1^2}{2c_{pг}} =
	<GasTemp> - \frac{{<C1>}^2}{2 \cdot <CompTurbineSpecificHeat>} = <T1> \/\ К$$
	\item Определим температуру конца адиабатного расширения:
	$$T_1^\prime = T_г - \frac{H_c}{c_{pг}} =
	<GasTemp> - \frac{<Hc>}{<CompTurbineSpecificHeat>} = <T1Prime> \/\ К$$
	\item Определим давление на выходе из СА:
	$$p_1 = p_г \left( \frac{T_1^\prime}{T_г} \right)^\frac{k_г}{k_г - 1} =
	<CompTurbinePIn> \cdot \left( \frac{<T1Prime>}{<GasTemp>} \right)^\frac{<CycleCompTurbineKGasShort>}{<CycleCompTurbineKGasShort> - 1} = <P1> \/\ МПа$$
	\item Определим плотность газа на выходе из СА:
	$$\rho_1 = \frac{p_1}{R_г T_1} =
	\frac{<P1>}{<CycleCompTurbineGasGasConstant> \cdot <T1>} = <Rho1> \/\ кг/м^3$$
	\item Зададим угол на выходе из СА:
	$$\alpha_1 = <Alpha1> \degree$$
	\item Определим осевую скорость на выходе из СА:
	$$c_{1a} = c_1 \cdot \sin \alpha_1 =
	{<C1> \cdot \sin<Alpha1>\degree} = <C1a> \/\ м/с$$
	\item Определим площадь на выходе из СА:
	$$A_1 = \frac{G}{c_{1a} \rho_1} =
	\frac{<G>}{<C1a> \cdot <Rho1>} = <A1> \/\ м^2$$
	\item Определим средний диаметр турбины на выходе из СА:
	$$D_1 = \sqrt{
		\frac{A_1}{\pi \left( \frac{l}{D} \right)_1}
	} = \sqrt{
		\frac{<A1>}{\pi \cdot <StatorLRel>}
	} = <StatorDOut> \/\ м $$
	\item Определим окружную скорость на среднем диаметре на входе в РК:
	$$u_1 = \frac{\pi D_1 n}{60} = \frac{\pi \cdot <RotorDIn> \cdot <RotationSpeed>}{60} = <U1> \/\ м/с$$
	\item Определим относительную скорость на входе в РК:
	$$w_1 = \sqrt{c_1^2 + u_1^2 - 2 c_1 u_1 \cos \alpha_1} =
	\sqrt{{<C1>}^2 + {<U1>}^2 - 2 \cdot <C1> \cdot <U1> \cdot \cos <Alpha1> \degree} = <W1> \/\ м/с$$
	\item Определим температуру торможения в относительном движении на входе в РК:
	$$T_{w1} = T_1 + \frac{w_1^2}{2c_{p г}} = <T1> + \frac{{<W1>}^2}{2 \cdot <CompTurbineSpecificHeat>} = <TW1> \/\ К$$
	\item Определим давление торможения в относительном движении на входе в РК:
	$$p_{w1} = p_1 \left( \frac{T_{w1}}{T_1} \right)^\frac{k_г}{k_г - 1} =
	 <P1> \cdot \left( \frac{<TW1>}{<T1>} \right)^\frac{<CycleCompTurbineKGasShort>}{<CycleCompTurbineKGasShort> - 1} = <PW1> \/\ МПа$$
	 \item Определим теплоперепад на РК:
	 $$H_л = H_т \rho \frac{T_1}{T_1^\prime} =
	 <Ht> \cdot <Reactivity> \cdot \frac{<T1>}{<T1Prime>} = <Hl> \/\ Дж/кг$$

	\item Определим расстояние в осевом направлении между выходными кромками лопаток СА и выходными кромками лопаток РК:
	 $$x = \frac{
	 	\frac{\delta_a}{ \left( \frac{l}{b_a} \right)_1 }	+
	 	\frac{1}{\left( \frac{l}{b_a} \right)_2 }
	 }{
	 	1 - \frac{\tan \gamma_п + \tan \gamma_в}
	 	{2 \left( \frac{l}{b_a} \right)_2}
	 } D_1 \left( \frac{l}{D} \right)_1 =
	 \frac{
	 	\frac{<StatorDeltaRel>}{ <StatorElongation> }	+
	 	<RotorElongation> }
	 {1 - \frac{\tan <GammaOut> + \tan <GammaIn>}
	 	{2 <RotorElongation>}
	 } <StatorDOut> \cdot <RotorLRel> = <X> \/\ м
	 $$
	 \item Определим средний диаметра на выходе из РК:
	 $$D_2 = D_1 + \frac{\tan \gamma_1 - \tan \gamma_2}{2} x =
	   		<StatorDOut> + \frac{\tan <GammaIn> \degree - \tan <GammaOut> \degree}{2} \cdot <X> =
	   		<RotorDOut> \/\ м$$
	 \item Определим длину лопатки на выходе из РК:
	 $$l_2 = D_1 \left( \frac{l}{D} \right)_1 + \frac{\tan \gamma_1 + \tan \gamma_2}{2} x =
	 		<StatorDOut> \cdot <StatorElongation> + \frac{\tan <GammaIn> \degree + \tan <GammaOut> \degree}{2} \cdot <X> =
	 		<L2> \/\ м$$
	 \item Определим относительную длину лопаток на выходе из РК:
	 $$\left( \frac{l}{D} \right)_2 = \frac{l_2}{D_2} = \frac{<L2>}{<RotorDOut>} =
	 <RotorLRel>$$

	 \item Определим окружную скорость на среднем диаметре на выходе из РК:
	 $$u_2 = \frac{\pi D_2 n}{60} = \frac{\pi \cdot <RotorDOut> \cdot <RotationSpeed>}{60} = <U2> \/\ м/с$$
	 \item Определим адиабатическую относительную скорость истечения газа из РК:
	 $$w_{2 ад} = \sqrt{w_1^2 + 2H_л +\left( u_2^2 - u_1^2 \right)} =
	 \sqrt{{<W1>}^2 + 2 \cdot <Hl> \cdot 10^6 +\left( {<U2>}^2 - {<U1>}^2 \right)} = <W2ad> \/\ м/с$$
	 \item Определим относительную скорость истечения газа из РК:
	 $$w_2 = \psi w_{2 ад} =
	 <Psi> \cdot <W2ad> = <W2> \/\ м/с$$
	 \item Определим статическую температуру на выходе из РК:
	 $$T_2 = T_1 + \frac{
	 	\left( w_1^2  - w_2^2 \right) + \left( u_2^2 - u_1^2 \right)
	 }{2 c_{p г}} =
	 <T1> + \frac{
	 	\left( {<W1>}^2  - {<W2>}^2 \right) + \left( {<U2>}^2 - {<U1>}^2 \right)
	 }{2 \cdot <CompTurbineSpecificHeat>} = <T2> \/\ К$$
	 \item Определим статическую температуру при адиабатическом процессе в РК:
	 $$T_2^\prime = T_1 + \frac{
	 	\left( w_1^2  - w_{2 ад}^2 \right) + \left( u_2^2 - u_1^2 \right)
	 }{2 c_{p г}} =
	 <T1> + \frac{
	 	\left( {<W1>}^2  - {<W2ad>}^2 \right) + \left( {<U2>}^2 - {<U1>}^2 \right)
	 }{2 \cdot <CompTurbineSpecificHeat>} = <T2Prime> \/\ К$$
	 \item Определим давление на выходе из РК:
	 $$p_2 = p_1 \left( \frac{T_2^\prime}{T_1} \right)^{\frac{k_г}{k_г - 1}} =
	 <P1> \left( \frac{<T2Prime>}{<T1>} \right)^{\frac{<CycleCompTurbineKGasShort>}{<CycleCompTurbineKGasShort> - 1}} = <P2> \/\ МПа$$
	 \item Определим угол в относительном движении на выходе из РК:
	 $$\beta_2 = \arcsin\frac{c_{2a}}{w_2} = \arcsin\frac{<C2a>}{<W2>} = <Beta2> \degree$$
	 \item Определим угол выхода из РК в абсолютном движении:
	 $$\alpha_2 = \arctan\frac{w_2 \cos \beta_2 - u_2}{c_{2a}} =
	 \arctan\frac{<W2> \cdot \cos <Beta2> \degree - <U2>}{<C2a>} = <Alpha2> \degree$$
	 \item Определим окружную составляющую скорости на выходе из РК:
	 $$c_{2u} = w_2 \cos \beta_2 - u_2 =
	 <W2> \cdot \cos <Beta2> \degree - <U2> = <C2u> \/\ м/с$$
	 \item Определим скорость потока на выходе из РК:
	 $$c_2 = \sqrt{c_{2u}^2 + c_{2a}^2} = \sqrt{{<C2u>}^2 + {<C2a>}^2} = <C2> \/\ м/с$$
	 \item Определим степень понижения давления в турбине компрессора:
	 $$\pi_{тк} = \frac{p_г}{p_2} = \frac{<CompTurbinePIn>}{<P2>} = <PiTC> $$
	 \item Определим осевую составляющую скорости газа за турбиной:
	 $$c_{2a} = c_2 \sin \alpha_2 = <C2> \sin <Alpha2> = <C2a> \/\ м/с$$
	 \item Опрдеелим плотность газа за турбиной:
	 $$\rho_2 = \frac{G}{c_2a \pi D_2 l_2} = \frac{<G>}{<C2a> \cdot \pi \cdot <RotorDOut> \cdot <L2>} = <Rho2> \/\ кг/м^3$$
	 \item Определим работу на окружности колеса:
	 $$L_u = c_{1u} u_1 + c_{2u} u_2 = <C1a> \cdot <U1> + <C2u> \cdot <U2> = <Lu> \cdot 10^6 \/\ Дж/кг$$
	 \item Определим КПД на окружности колеса:
	 $$\eta_u = \frac{L_u}{H_t} = \frac{<Lu>}{<Ht>} = <EtaU> $$
	 \item Определим удельные потери на статоре:
	 $$h_c = \left( \frac{1}{\phi^2} - 1 \right) \frac{c_1^2}{2} =
	 \left( \frac{1}{{<Phi>}^2} - 1 \right) \frac{{<C1>}^2}{2} = <hs> \cdot 10^3 \/\ Дж/кг$$
	 \item Определим удельные потери на роторе:
	 $$h_р = \left( \frac{1}{\psi^2} - 1 \right) \frac{w_2^2}{2} =
	 \left( \frac{1}{{<Psi>}^2} - 1 \right) \frac{{<W2>}^2}{2} = <hr> \cdot 10^3 \/\ Дж/кг$$
	 \item Определим удельные потери с выходной скоростью:
	 $$h_{вых} = \frac{c_2^2}{2}= \frac{{<C2>}^2}{2} = <hOut> \cdot 10^3 \/\ Дж/кг$$
	 \item Определим удельные потери в радиальном зазоре:
	 $$h_з = 1.37 \cdot \left( 1 + 1.6 \rho \right)
	 \left[ 1 + \left( \frac{l}{D} \right)_1 \right] \frac{\delta_r}{l_2} L_u = $$
	 $$ = 1.37 \cdot \left( 1 + 1.6 \cdot <Reactivity> \right)
	 \left[ 1 + <RotorLRel> \right] \frac{<DeltaR>}{<L2>} \cdot <Lu> =
	 <hRadial> \cdot 10^3 \/\ Дж/кг$$
	 \item Определим удельные потери на вентиляцию:
	 $$h_{вент} = 1.07 D_2^2 \left( \frac{u_2}{100} \right)^3 \rho_2 \cdot 1000 =
	 1.07 \cdot {<RotorDOut>}^2 \left( \frac{<U2>}{100} \right)^3 \cdot <Rho2> \cdot 1000 = <hVent> \cdot 10^3 \/\ Дж/кг$$
	 \item Определим температуру торможения за РК:
	 $$T_2^* = T_2 + \frac{h_з + h_{вент} + h_{вых}}{c_{pг}} =
	 <T2> + \frac{<hRadial> \cdot 10^3 + <hVent> \cdot 10^3 + <hOut> \cdot 10^3}{<CompTurbineSpecificHeat>} = <T2Stag> \/\ К$$
	 \item Определим давление торможения за РК:
	 $$p_2^* = p_2 \left( \frac{T_2^*}{T_2} \right)^{\frac{k_г}{k_г - 1}} =
	 <P2> \cdot \left( \frac{<T2Stag>}{<T2>} \right)^{\frac{<CycleCompTurbineKGasShort>}{<CycleCompTurbineKGasShort> - 1}} = <P2Stag> \/\ МПа$$
	 \item Определим мощностной КПД турбины:
	 $$\eta_{т \/\ мощн} = \eta_u - \frac{h_з + h_{вент}}{H_т} =
	 <EtaU> - \frac{<hRadial> \cdot 10^3 + <hVent> \cdot 10^3}{<Ht> \cdot 10^6} = <EtaPower>$$
	 \item Определим работу турбины:
	 $$L_т = H_т \eta_т = <Ht> \cdot 10^6 \cdot <EtaPower> = <Lt> \cdot 10^6 \/\ Дж/кг$$
	 \item Определим теплоперепад по параметрам торможения:
	 $$H_т^* = c_{pг} T_г \left[ 1 - \left( \frac{p_2^*}{p_г^*} \right)^\frac{k_г - 1}{k_г} \right] =
	 <CompTurbineSpecificHeat> \cdot <GasTemp> \left[ 1 - \left( \frac{<P2Stag>}{<CompTurbinePIn>} \right)^\frac{<CycleCompTurbineKGasShort> - 1}{<CycleCompTurbineKGasShort>} \right] = <HtStag> \cdot 10^6 \/\ Дж/кг $$
	 \item Определим КПД турбины по параметрам торможения:
	 $$\eta_т = \frac{L_т}{H_т^*} =
	 \frac{<Lt> \cdot 10^6}{<HtStag> \cdot 10^6} = <EtaT>$$

\end{enumerate}